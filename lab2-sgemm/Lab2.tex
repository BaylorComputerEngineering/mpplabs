\documentclass{article}

\title{Lab 2}
\author{Keith Evan Schubert}

\begin{document}
\maketitle

\section{Objective}
The purpose of this lab is to implement a basic dense matrix multiplication routine.

\section{Activity}

\begin{enumerate}
\item Login to kodiak.  cd to your mpplabs directory and type \verb1git pull1.
\item Edit the file \verb1<lab-directory>/main.cu1 to implement the following where indicated:
	\begin{enumerate}
	\item Allocate device memory
	\item Copy host memory to device
	\item Copy results from device to host
	\item Free device memory
	\end{enumerate}	 
\item Edit the file \verb1<lab-directory>/kernel.cu1 to initialize the thread block and kernel grid dimensions and invoke the CUDA kernel, and to implement the matrix multiplication kernel code..
\item Compile and test your code.  
\begin{verbatim}
	cd <lab-directory>
	make
	nano sgemm.sh  # add sgemm commands per below
	    ~/<lab-directory>/sgemm			# Uses the default matrix sizes
	    ~/<lab-directory>/sgemm <m>			# Uses square m x m matrices
	    ~/<lab-directory>/sgemm <m> <k> <n>	# Uses (m x k) and 
	                                  # (k x n) input matrices
	qsub -q tardis sgemm.sh
\end{verbatim}
\end{enumerate}

\section{Turn in}
Upload to the course Canvas site:
\begin{enumerate}
\item a report that includes :
	\begin{enumerate}
	\item the output 
	\item analysis of the performance: Try the code for several sizes, square and non-square matrices, and matrices that fit and don't fit (neatly) in the blocks  How does the time change?  Does each part change the same?
	\item answer section where you answer the following:
		\begin{enumerate}
		\item How many times is each element of each input matrix loaded during the execution of the kernel?
		\item What is the memory-access to floating-point computation ratio in each thread? Consider multiplication and addition as separate operations, and ignore the global memory store at the end. Only count global memory loads towards your off-chip bandwidth.
		\end{enumerate}
	\end{enumerate}
\item main.cu
\item kernel.cu
\end{enumerate} 
The cuda code will be graded for completeness, correctness, handling of boundary, and style (5pts).  The report will be graded on readability, clarity, analysis, and solution to the questions (5pts).


\section{Going Further}
We will be looking at some areas soon, so think about how you could group the sections to minimize memory loads.

\end{document}